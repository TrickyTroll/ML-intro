%% Generated by Sphinx.
\def\sphinxdocclass{report}
\documentclass[letterpaper,10pt,french]{sphinxmanual}
\ifdefined\pdfpxdimen
   \let\sphinxpxdimen\pdfpxdimen\else\newdimen\sphinxpxdimen
\fi \sphinxpxdimen=.75bp\relax

\PassOptionsToPackage{warn}{textcomp}
\usepackage[utf8]{inputenc}
\ifdefined\DeclareUnicodeCharacter
% support both utf8 and utf8x syntaxes
  \ifdefined\DeclareUnicodeCharacterAsOptional
    \def\sphinxDUC#1{\DeclareUnicodeCharacter{"#1}}
  \else
    \let\sphinxDUC\DeclareUnicodeCharacter
  \fi
  \sphinxDUC{00A0}{\nobreakspace}
  \sphinxDUC{2500}{\sphinxunichar{2500}}
  \sphinxDUC{2502}{\sphinxunichar{2502}}
  \sphinxDUC{2514}{\sphinxunichar{2514}}
  \sphinxDUC{251C}{\sphinxunichar{251C}}
  \sphinxDUC{2572}{\textbackslash}
\fi
\usepackage{cmap}
\usepackage[T1]{fontenc}
\usepackage{amsmath,amssymb,amstext}
\usepackage{babel}



\usepackage{times}
\expandafter\ifx\csname T@LGR\endcsname\relax
\else
% LGR was declared as font encoding
  \substitutefont{LGR}{\rmdefault}{cmr}
  \substitutefont{LGR}{\sfdefault}{cmss}
  \substitutefont{LGR}{\ttdefault}{cmtt}
\fi
\expandafter\ifx\csname T@X2\endcsname\relax
  \expandafter\ifx\csname T@T2A\endcsname\relax
  \else
  % T2A was declared as font encoding
    \substitutefont{T2A}{\rmdefault}{cmr}
    \substitutefont{T2A}{\sfdefault}{cmss}
    \substitutefont{T2A}{\ttdefault}{cmtt}
  \fi
\else
% X2 was declared as font encoding
  \substitutefont{X2}{\rmdefault}{cmr}
  \substitutefont{X2}{\sfdefault}{cmss}
  \substitutefont{X2}{\ttdefault}{cmtt}
\fi


\usepackage[Sonny]{fncychap}
\ChNameVar{\Large\normalfont\sffamily}
\ChTitleVar{\Large\normalfont\sffamily}
\usepackage[,numfigreset=1,mathnumfig]{sphinx}

\fvset{fontsize=\small}
\usepackage{geometry}


% Include hyperref last.
\usepackage{hyperref}
% Fix anchor placement for figures with captions.
\usepackage{hypcap}% it must be loaded after hyperref.
% Set up styles of URL: it should be placed after hyperref.
\urlstyle{same}


\usepackage{sphinxmessages}




\title{Introduction à la reconnaissance optique de caractère}
\date{sept. 29, 2020}
\release{}
\author{Émile Bergeron, Samuel Paquin, Étienne Parent et Jérémie Sanfaçon}
\newcommand{\sphinxlogo}{\vbox{}}
\renewcommand{\releasename}{}
\makeindex
\begin{document}

\ifdefined\shorthandoff
  \ifnum\catcode`\=\string=\active\shorthandoff{=}\fi
  \ifnum\catcode`\"=\active\shorthandoff{"}\fi
\fi

\pagestyle{empty}
\sphinxmaketitle
\pagestyle{plain}
\sphinxtableofcontents
\pagestyle{normal}
\phantomsection\label{\detokenize{intro::doc}}


This is a small sample book to give you a feel for how book content is
structured.

Check out the content pages bundled with this sample book to get started.


\chapter{Rapport préliminaire}
\label{\detokenize{rapport_preliminaire:rapport-preliminaire}}\label{\detokenize{rapport_preliminaire::doc}}
Cette section contient la planification de notre projet ainsi que les
résultats de nos premières recherches.


\section{Énoncé du sujet}
\label{\detokenize{enonce_sujet:enonce-du-sujet}}\label{\detokenize{enonce_sujet::doc}}

\subsection{Intro JSanf}
\label{\detokenize{enonce_sujet:intro-jsanf}}
L’intelligence artificielle est un sujet d’actualité de haute importance avec les
nombreuses avancées
technologiques qui se font depuis les dernières années avec des avancés comme la
conduite automatique
par la compagnie Tesla ainsi que le Neuralink. L’intelligence artificielle ou
l’IA nous entourent
dans notre vie que se soit sur notre cellulaire, sur notre ordinateur et même nos automobile.
Quel est le fonctionnent de ce processus et comment peut\sphinxhyphen{}il être utilisé afin de
bénéficier l’être humain?. Pour tenter de répondre à cet question, nous allons
écrire un programme qui permet de faire de l” Optical Character Recognition ou OCR
qui consiste à lire un chiffre écrit à la main et retourner une réponse grâce à
un entraînement sur l’ordinateur nommé Supervised Learning. Pour en découvrir
plus sur ce processus, nous allons aussi voir l’impact d’un entraînement varié
sur le modèle en changeant l’écriture de l’humain pour voir si cela constitue un
biais.


\section{Description de la manière dont on entend traiter le sujet}
\label{\detokenize{description:description-de-la-maniere-dont-on-entend-traiter-le-sujet}}\label{\detokenize{description::doc}}
Vous devez décrire comment cette question a généralement été traitée jusqu’à
maintenant et spécifier les angles d’approche que vous choisirez pour répondre
à votre question. Vous devez indiquer les sources au
fil du texte, tel que spécifié dans la méthode APA des normes de présentations
du Cégep de Sainte\sphinxhyphen{}Foy. Vous devez également formuler une hypothèse, c’est\sphinxhyphen{}à\sphinxhyphen{}dire
une réponse provisoire à votre question. C’est
cette hypothèse que vous cherchez à clarifier.
Votre description doit inclure chacune des dimensions suivantes :
\begin{itemize}
\item {} 
mathématiques

\item {} 
au moins une des disciplines : biologie, chimie ou physique,

\item {} 
liens entre la science, la technologie et l’évolution de la société.

\end{itemize}

Question: Quel est le fonctionnent de ce processus et comment peut\sphinxhyphen{}il être
utilisé afin de bénéficier l’être humain?

Pour répondre à cette question, nous étudions le code derrière cette technologie.
En comprenant le processus de programmation, les concepts mathématiques s’intègrent
naturellement à la démarche, permettant une bonne compréhension globale de
cette technologie. Par exemple, en apprenant les interactions entre les couches
de neurones, l’algèbre linéaire requis pour faire des opérations sur les matrices
d’intrants et de sortants des neurones découle logiquement. De plus,
nous considérerons les applications concrètes de cette technologie, ainsi que les
ramifications potentielles de l’apprentissage machine dans le futur, et comment
celles\sphinxhyphen{}ci pourront améliorer la société.


\section{Plan de rédaction du travail}
\label{\detokenize{plan:plan-de-redaction-du-travail}}\label{\detokenize{plan::doc}}
Nous allons répondre à notre question sous plusieurs aspects;


\subsection{Recherche des bénéfices de la reconnaissance de caractères}
\label{\detokenize{plan:recherche-des-benefices-de-la-reconnaissance-de-caracteres}}\begin{itemize}
\item {} 
Recherche et documentation sur l’histoire de la technologie ainsi que ses
utilités dans le passé.

\item {} 
Recherche sur les avancées importante menées par l’IA

\item {} 
Recherche sur les avancées en ce moment

\end{itemize}


\subsection{Recherche sur le contionnement de l’OCR}
\label{\detokenize{plan:recherche-sur-le-contionnement-de-l-ocr}}\begin{itemize}
\item {} 
Recherche et Documentation sur le fonctionnement de l’intelligence Artificielle

\item {} 
Recherche et Documentation sur l’OCR

\item {} 
Recherche et Documentation sur la programmation d’un OCR

\item {} 
Programmation d’un programme qui fait de l’OCR

\item {} 
Recherche et Documentation sur l’entraînement d’un OCR

\item {} 
Étude et Documentation sur l’utilisation de plusieurs entraînements différent
au programme en utilisant des humains différent ainsi qu’un type d’écriture
différent.

\end{itemize}


\section{Énoncé du sujet}
\label{\detokenize{biblio_commented:enonce-du-sujet}}\label{\detokenize{biblio_commented::doc}}






\renewcommand{\indexname}{Index}
\printindex
\end{document}