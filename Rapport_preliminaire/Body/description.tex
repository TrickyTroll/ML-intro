Nous supposons que l’intelligence artificielle, le principe d’associer une certaine intelligence humaine à un programme est le futur du domaine de la technologie et est donc bénéfique dans de nombreuses sphères. En effet,  au niveau économique en augmentant la productivité et l’efficacité, politique en s’occupant de l’analyse de données, du transport en innovant des voitures autonomes, de la santé en accélérant l’analyse de données, et plusieurs autres sphères de la vie humaine (https://www.economie.gouv.qc.ca/objectifs/informer/vecteurs/vecteurs-actualites/vecteurs-actualites-details/?no_cache=1&tx_ttnews%5Btt_news%5D=23153&tx_ttnews%5Bcat%5D=&cHash=2f4aa33c55d12596ff0c2d6f468960bc ) Les avancées en intelligence artificielle peuvent cependant devenir néfaste lors de certaines situations. Par exemple, l’automatisation de tâches manuelles, qui peut avoir lieu grâce à l’intelligence artificielle,  mène à une augmentation du taux de chômage. Donc, il est important de bien utilisé cette technologie et par ce fait, il faut débuter par la comprendre. C’est pourquoi 
nous allons expliquer le code derrière cette technologie. En comprenant le processus de programmation, les concepts mathématiques s'intègrent naturellement à la démarche, permettant une bonne compréhension globale de cette technologie. Par exemple, en apprenant les interactions entre les couches de neurones, l'algèbre linéaire requis pour faire des opérations sur les matrices d'intrants et de sortants des neurones découle logiquement. De plus, nous considérerons les applications concrètes de cette technologie, ainsi que les ramifications potentielles de l'apprentissage machine dans le futur, et comment celles-ci pourront améliorer la société.

