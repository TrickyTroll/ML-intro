\subsection{description de la manière dont on entend traiter le sujet}
Vous devez décrire comment cette question a généralement été traitée jusqu’à
maintenant et spécifier les angles d’approche que vous choisirez pour répondre
à votre question. Vous devez indiquer les sources au
fil du texte, tel que spécifié dans la méthode APA des normes de présentations
du Cégep de Sainte-Foy. Vous devez également formuler une hypothèse, c’est-à-dire
une réponse provisoire à votre question. C’est
cette hypothèse que vous cherchez à clarifier.
Votre description doit inclure chacune des dimensions suivantes :
\begin{itemize}
  \item mathématiques,
  \item au moins une des disciplines : biologie, chimie ou physique,
  \item liens entre la science, la technologie et l’évolution de la société.
\end{itemize}

Question: Quel est le fonctionnent de ce processus et comment peut-il être utilisé afin de bénéficier l’être humain?

Pour répondre à cette question, nous étudions le code derrière cette technologie. 
En comprenant le processus de programmation, les concepts mathématiques s'intègrent
naturellement à la démarche, permettant une bonne compréhension globale de
cette technologie. Par exemple, en apprenant les interactions entre les couches
de neurones, l'algèbre linéaire requis pour faire des opérations sur les matrices
d'intrants et de sortants des neurones découle logiquement. De plus,
nous considérerons les applications concrètes de cette technologie, ainsi que les
ramifications potentielles de l'apprentissage machine dans le futur, et comment
celles-ci pourront améliorer la société.
