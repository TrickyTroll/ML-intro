\documentclass[12pt,letterpaper]{article}
\usepackage{pdfpages}
\usepackage[french]{babel}
\usepackage[T1]{fontenc}
\usepackage[utf8]{inputenc}
\usepackage{fancyhdr}
\usepackage{enumitem}
\usepackage{graphicx}
\usepackage{amsmath}
\usepackage{siunitx}
\usepackage{hyperref}
\usepackage{wrapfig}
\usepackage{booktabs}
\usepackage{rotating}
\usepackage{gensymb}
\usepackage{lscape}
\DeclareMathSymbol{.}{\mathord}{letters}{"3B}
\begin{document} % Le document commence après cette ligne.
\documentclass[12pt,letterpaper]{report}
\usepackage{pdfpages}
\usepackage[french]{babel}
\usepackage[T1]{fontenc}
\usepackage[utf8]{inputenc}
\usepackage{fancyhdr}
\usepackage{enumitem}
\usepackage{graphicx}
\usepackage{amsmath}
\usepackage{siunitx}
\usepackage{hyperref}
\usepackage{wrapfig}
\usepackage{booktabs}
\usepackage{rotating}
\usepackage{gensymb}
\usepackage{lscape}
\DeclareMathSymbol{.}{\mathord}{letters}{"3B}
\begin{document}
\begin{titlepage}
    \noindent
    Étienne PARENT\\
    Samuel PAQUIN \\
    Émile BERGERON\\
    Jérémie SANFAÇON\\
    \noindent
    Groupe 0001


    \indent
	\centering
	\vspace{2cm}

	{\scshape\textbf{ACTIVITÉ D'INTÉGRATION}}\\
	{\scshape\textbf{Rapport préliminaire}\par}
	\vspace{2cm}
	{Travail présenté à M.  Louis Bérubé}\\
	\vspace{2cm}
	{Activité d'intégration en mathématiques}\\
    {201-FZZ-03}
	\vfill
	{Département de Mathématiques}\\
	{Sciences de la nature}\\
	{Cégep de Sainte-Foy}\\
	{\large \today}
\end{titlepage}
\end{document}
 % Ajout de la page de présentation.
\setlength{\parindent}{0in}
\pagestyle{fancy}
\fancyhf{}
\fancyfoot[C]{\thepage}
\pagenumbering{roman}
\tableofcontents
\newpage
% Le body du document commence ici.
\pagenumbering{arabic}
% Section sur l'énoncé du sujet. La section devrait être écrite dans le fichier
% enonce_sujet.tex sous Components.
\section{Énoncé du sujet}
Énoncé du sujet formulé comme une problématique de nature scientifique. Pour ce faire, vous devez
d’abord situer votre sujet. Vous devez expliquer comment vous en êtes venus à votre problème ou encore
quels événements ou opinions généralement partagées vous mènent à votre problème. Vous devez ensuite
formuler une question claire, pertinente et qui permet une prise de position. Cette question est la pierre
de touche de votre travail. Il est important qu’elle soit bien formulée. 

L'intelligence artificielle est un sujet d'actualité de haute importance avec les nombreuses avancées 
technologiques qui se font depuis les dernières années avec des avancés comme la conduite automatique par la compagnie Tesla ainsi que le Neuralink. L'intelligence artificielle ou l 'IA nous entourent dans notre vie que se soit sur notre cellulaire, sur notre ordinateur
et même nos automobile. Quel est le fonctionnent de ce processus et comment peut-il être utilisé afin de bénéficier l’être humain?. 
Pour tenter de répondre à cet question, nous allons écrire un programme qui permet de faire de l' Optical Character Recognition ou OCR 
qui consiste à lire un chiffre écrit à la main et retourner une réponse grâce à un entraînement sur l'ordinateur 
nommé "Supervised Learning". Pour en découvrir plus sur ce processus, nous allons aussi voir l'impact d'un 
entraînement varié sur le modèle en changeant l'écriture de l'humain pour voir si cela constitue un biais.


% Section sur la description. Devrait être faite dans description.tex.
\section{Description de la manière dont on entend traiter le sujet}
Nous supposons que l’intelligence artificielle, le principe d’associer une certaine intelligence humaine à un programme est le futur du domaine de la technologie et est donc bénéfique dans de nombreuses sphères. En effet,  au niveau économique en augmentant la productivité et l’efficacité, politique en s’occupant de l’analyse de données, du transport en innovant des voitures autonomes, de la santé en accélérant l’analyse de données, et plusieurs autres sphères de la vie humaine (https://www.economie.gouv.qc.ca/objectifs/informer/vecteurs/vecteurs-actualites/vecteurs-actualites-details/?no_cache=1&tx_ttnews%5Btt_news%5D=23153&tx_ttnews%5Bcat%5D=&cHash=2f4aa33c55d12596ff0c2d6f468960bc ) Les avancées en intelligence artificielle peuvent cependant devenir néfaste lors de certaines situations. Par exemple, l’automatisation de tâches manuelles, qui peut avoir lieu grâce à l’intelligence artificielle,  mène à une augmentation du taux de chômage. Donc, il est important de bien utilisé cette technologie et par ce fait, il faut débuter par la comprendre. C’est pourquoi 
nous allons expliquer le code derrière cette technologie. En comprenant le processus de programmation, les concepts mathématiques s'intègrent naturellement à la démarche, permettant une bonne compréhension globale de cette technologie. Par exemple, en apprenant les interactions entre les couches de neurones, l'algèbre linéaire requis pour faire des opérations sur les matrices d'intrants et de sortants des neurones découle logiquement. De plus, nous considérerons les applications concrètes de cette technologie, ainsi que les ramifications potentielles de l'apprentissage machine dans le futur, et comment celles-ci pourront améliorer la société.


% Section sur le plan de rédaction. Devrait être fait dans le fichier plan.tex.
\section{Plan de rédaction du travail}
L'Intelligence Artificielle à vraiment exploser avec le début du 21ieme siècle.

% Bibliographie générée par bibTeX. Les références devraient être dans
% bibliography.bib sous Bibliography.
\bibliography{./Bibliography/bibliography.bib}

% Rien après cette ligne.
\end{document}
