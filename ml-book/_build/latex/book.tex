%% Generated by Sphinx.
\def\sphinxdocclass{report}
\documentclass[letterpaper,10pt,french]{sphinxmanual}
\ifdefined\pdfpxdimen
   \let\sphinxpxdimen\pdfpxdimen\else\newdimen\sphinxpxdimen
\fi \sphinxpxdimen=.75bp\relax

\PassOptionsToPackage{warn}{textcomp}
\usepackage[utf8]{inputenc}
\ifdefined\DeclareUnicodeCharacter
% support both utf8 and utf8x syntaxes
  \ifdefined\DeclareUnicodeCharacterAsOptional
    \def\sphinxDUC#1{\DeclareUnicodeCharacter{"#1}}
  \else
    \let\sphinxDUC\DeclareUnicodeCharacter
  \fi
  \sphinxDUC{00A0}{\nobreakspace}
  \sphinxDUC{2500}{\sphinxunichar{2500}}
  \sphinxDUC{2502}{\sphinxunichar{2502}}
  \sphinxDUC{2514}{\sphinxunichar{2514}}
  \sphinxDUC{251C}{\sphinxunichar{251C}}
  \sphinxDUC{2572}{\textbackslash}
\fi
\usepackage{cmap}
\usepackage[T1]{fontenc}
\usepackage{amsmath,amssymb,amstext}
\usepackage{babel}



\usepackage{times}
\expandafter\ifx\csname T@LGR\endcsname\relax
\else
% LGR was declared as font encoding
  \substitutefont{LGR}{\rmdefault}{cmr}
  \substitutefont{LGR}{\sfdefault}{cmss}
  \substitutefont{LGR}{\ttdefault}{cmtt}
\fi
\expandafter\ifx\csname T@X2\endcsname\relax
  \expandafter\ifx\csname T@T2A\endcsname\relax
  \else
  % T2A was declared as font encoding
    \substitutefont{T2A}{\rmdefault}{cmr}
    \substitutefont{T2A}{\sfdefault}{cmss}
    \substitutefont{T2A}{\ttdefault}{cmtt}
  \fi
\else
% X2 was declared as font encoding
  \substitutefont{X2}{\rmdefault}{cmr}
  \substitutefont{X2}{\sfdefault}{cmss}
  \substitutefont{X2}{\ttdefault}{cmtt}
\fi


\usepackage[Sonny]{fncychap}
\ChNameVar{\Large\normalfont\sffamily}
\ChTitleVar{\Large\normalfont\sffamily}
\usepackage[,numfigreset=1,mathnumfig]{sphinx}

\fvset{fontsize=\small}
\usepackage{geometry}


% Include hyperref last.
\usepackage{hyperref}
% Fix anchor placement for figures with captions.
\usepackage{hypcap}% it must be loaded after hyperref.
% Set up styles of URL: it should be placed after hyperref.
\urlstyle{same}


\usepackage{sphinxmessages}




\title{Introduction à la reconnaissance optique de caractère}
\date{oct. 02, 2020}
\release{}
\author{Émile Bergeron, Samuel Paquin, Étienne Parent, Jérémie Sanfaçon}
\newcommand{\sphinxlogo}{\vbox{}}
\renewcommand{\releasename}{}
\makeindex
\begin{document}

\ifdefined\shorthandoff
  \ifnum\catcode`\=\string=\active\shorthandoff{=}\fi
  \ifnum\catcode`\"=\active\shorthandoff{"}\fi
\fi

\pagestyle{empty}
\sphinxmaketitle
\pagestyle{plain}
\sphinxtableofcontents
\pagestyle{normal}
\phantomsection\label{\detokenize{intro::doc}}


La documentation sur l’utilisation de ce site n’a pas encore été faite. :)


\chapter{Rapport préliminaire}
\label{\detokenize{rapport_preliminaire:rapport-preliminaire}}\label{\detokenize{rapport_preliminaire::doc}}
Cette section contient la planification de notre projet ainsi que les
résultats de nos premières recherches.


\section{Énoncé du sujet}
\label{\detokenize{enonce_sujet:enonce-du-sujet}}\label{\detokenize{enonce_sujet::doc}}

\subsection{Mise en contexte}
\label{\detokenize{enonce_sujet:mise-en-contexte}}
L’intelligence artificielle est au coeur de l’actualité depuis près d’une
décénie. Elle est dejà entrain de changer le monde , et ce, dans plusieurs
secteurs incluant la finance, la sécurité, la santé, la justice criminel,
les moyens de transports, la publicité, et plusieurs autres.

Que ça soit des décisions sur l’investissement d’un portefeuille
d’un individu ou de la détection de fraude en identifiant des anormalités, l’intelligence
artificielle est de plus en plus présente dans le secteur de la finance.

Du côté de la
sécurité, un excellent exemple serait \sphinxhref{https://en.wikipedia.org/wiki/Project\_Maven}{Project Maven}
un projet d” intelligence artificielle du \sphinxhref{https://en.wikipedia.org/wiki/The\_Pentagon}{Pentagon}
des États\sphinxhyphen{}Unis qui est capable de passer à travers plusieurs informations,
vidéos et photos pour détecter des dangers potentiels.

L’intelligence artificielle est très importante dans la santé avec des compagnies comme
\sphinxhref{https://www.merantix.com/}{Merantix}, une compagnie Allemande qui on permis de detecté
des ganglions lymphatiques ainsi que des problèmes liés à ceux\sphinxhyphen{}ci tel que des lésions
ou des cancers. L’étude de séquence d’ADN par l’intelligence artificielle permet de détecter
des maladies génétiques et des cancers.

Un des domaine le plus importants en ce moment serait, les moyens de transport avec plus de \$80
milliard investi dans des véhicule de conduite autonome entre 2014 et 2017. L’intelligence
artficielle dans ce domaine aurait pour but de diminuer grandement l’erreur humaine dans les transports
et réduire à presque zéro les accidents si la majorité des autos était intelligente. De plus, cela réduirait
aussi grandement le traffic grâce à la communication entre les automobiles intelligents. La compagnie \sphinxhref{https://www.tesla.com/}{Tesla}
en est déjà très avancée pour ce qui est de leur auto intelligente.

Comme on peut le voir, cette technologie a permis de multiples avancées dans des domaines où
il se fait extrêmement difficile de modéliser la problématique selon une
fonction mathématique particulière. L’analyse de language en est un bon exemple.
Le travail ne peut être modélisé par une seule fonction mathématique puisque
les conditions souvent changeantes nécessiteraient une multitude de fonctions
différentes pour chaque environnement qui n’est pas réaliste. La solution est
plutôt « d’entraîner » un ordinateur à comprendre le monde qui l’entoure.
Pour continuer avec l’exemple de l’analyse du language, une solution serait
de fournir à l’ordinateur une immense quantité d’exemples et de solutions afin
qu’il développe la capacité de prédire la solution à de nouveaux exemples.
\sphinxhref{https://github.com/openai/gpt-3}{GPT\sphinxhyphen{}3},
un nouveau modèle d’intelligence artificielle produit par
\sphinxhref{https://openai.com}{OpenAI}, a permis à des développeurs de créer un programme
lui même capable de programmer à partir de demandes spécifiques faites par un
utilisateur.


\subsection{Le début de la découverte des inconvénients}
\label{\detokenize{enonce_sujet:le-debut-de-la-decouverte-des-inconvenients}}
Malgré les avancées incroyables que l’intelligence artificielle a déjà permis et
continuera de permettre dans le futur, elle n’est pas sans ses inconvénients.


\subsection{Le biais}
\label{\detokenize{enonce_sujet:le-biais}}
Au
courant des dernières années, les systèmes intelligents sont de plus en plus
reconnus coupables de discrimination envers certains groupes d’individus. Une
étude réalisée par le \sphinxhref{https://www.nist.gov/}{NIST} à étudié le taux d’erreur de
différents programmes de reconnaissance faciale en fonction des différences de
sexe et d’ethnicité des individus sur les photos analysées. L’étude
présente des taux d’erreur
jusqu’à cent fois plus élevés pour des personnes d’origine asiatique ou
africaine lorsque comparé à des personnes d’origine européenne \sphinxcite{biblio_commented:nistbias}.
Le taux d’erreur est aussi plus élevé chez les femmes que chez les hommes, et
ce, peut importe l’origine.

Un autre résultat important de cette étude est que le taux d’erreur associé à la
reconnaissance de personnes asiatiques n’est pas présent dans des programmes
réalisé dans des pays d’Asie. Cette observation permet de déduire l’un des plus
grands problèmes liés à l’intelligence artificielle: le biais.

Contrairement à une fonction mathématique qui transforme un chiffre de manière
définie, les procédés menant à la reconnaissance faciale sont beaucoup plus
flous et souvent très mal compris. Plusieurs considèrent les programmes
entraînés comme des « boîtes noires ». Il est difficile de prédire ce qui sortira
de la boîte lorsque l’on y insère quelque chose, et il est encore plus difficile
de comprendre pourquoi le programme prend certaines décisions plus que d’autres.

Cette imprévisibilité inquiète plusieurs. Elle rend la tâche de corriger le
biais assez ardue. Elle fait aussi en sorte qu’il est difficile de prédire le
comportement du programme dans des cas extrêmes sans avoir à lui faire passer
des tests dans ces conditions. Le biais est donc un phénomène difficile à
corriger, ce qui entraîne des questionnements en rapport aux bienfaits de
l’utilisation de l’intelligence artificielle.

Certaines régions du monde
commencent à banir l’utilisation de la reconnaissance faciale par les forces
de l’ordre. C’est le cas de la ville de Portland, en Oregon \sphinxcite{biblio_commented:cnnportland}.
La ville a décidé de bannir l’utilisation de la technologie suite à des craintes
en liées à son manque de précision, surtout lorsqu’utilisée sur des individus
appartenant à une minorité visible.

\begin{figure}[htbp]
\centering
\capstart

\noindent\sphinxincludegraphics{{black_box}.png}
\caption{L’analogie de la boîte noire.}\label{\detokenize{enonce_sujet:boite-noire}}\end{figure}


\subsection{Une deuxième révolution industrielle}
\label{\detokenize{enonce_sujet:une-deuxieme-revolution-industrielle}}
Une autre inquiétude liée à l’intelligence artificielle est l’importante
quantité d’emplois qui risque de disparaître puisqu’ils seront maintenant
occupés par des ordinateurs. Ces inquiétudes sont justifiées. Plusieurs articles,
dont
\sphinxhref{https://www.cnbc.com/2019/01/14/the-oracle-of-ai-these-kinds-of-jobs-will-not-be-replaced-by-robots-.html}{celui\sphinxhyphen{}ci}
publié par CNN ainsi que
\sphinxhref{https://medium.com/@ChanPriya/15-jobs-that-will-never-be-replaced-by-ai-512bfbbed0d6}{cette publication}
sur Medium tentent de rassurer la population en mentionnant des emplois qui ne
pourraient apparement jamais être remplacés par des ordinateurs. Ils mentionnent
entre autres les emplois créatifs, accompagnés des emplois nécessitant beaucoup
d’intéractions humaines.

Pourtant, le domaine de l’IA avance à chaque année, et il existe maintenant une
panoplie de programmes capable de
\sphinxhref{https://openai.com/blog/musenet/}{composer de la musique},
\sphinxhref{https://www.nvidia.com/en-us/research/ai-playground/}{maîtriser les arts visuels}
ainsi qu”\sphinxhref{https://www.youtube.com/watch?v=D5VN56jQMWM}{entretenir des conversations au téléphone}.

\begin{figure}[htbp]
\centering
\capstart

\noindent\sphinxincludegraphics{{duplex}.jpeg}
\caption{Le PDG de Google présentant une démonstation de Google Duplex.}\label{\detokenize{enonce_sujet:duplex-presentation}}\end{figure}

Il est dangereux d’extrapoler le progrès qui a été fait au courant des dernières
années sur les décénies à venir. Certaines lois limitant le développement de
l’IA, ou des limitations physiques au présent rythome d’augmentation de la
puissance de calcul des ordinateurs pourraient survenir grandement ralentir
le développement de la technologie. Si nous tentons tout de même de le faire,
les inquiétudes vécues par plusieurs semblent raisonnables.


\subsection{Comprendre la technologie pour démistifier les inquiétudes}
\label{\detokenize{enonce_sujet:comprendre-la-technologie-pour-demistifier-les-inquietudes}}
Bien que les précédentes inquiétudes façe à l’intelligence artificielle soient
totalement justifiées, elles ne sont pas sans solution. Si son développement
est fait de manière éthique et s’il est bien encâdré, nous pourrions en retirer
plus d’avantages que d’inconvénient. Pour bien comprendre les inquiétudes, il
faut d’abord comprendre les enjeux. C’est pourquoi nous tenterons de répondre
à la question suivante.

\sphinxstyleemphasis{Quel est le fonctionnent de l’intelligence artificielle et comment devrait\sphinxhyphen{}elle
être utilisée afin de bénéficier l’être humain?}


\section{Description de la manière dont on entend traiter le sujet}
\label{\detokenize{description:description-de-la-maniere-dont-on-entend-traiter-le-sujet}}\label{\detokenize{description::doc}}
Nous croyons que l’humain gagne à utiliser l’intelligence artificielle, mais que son application a des limites et qu’il est important de l’utiliser avec discernement. Pour ce faire, il faut débuter par bien la comprendre. Nous allons donc expliquer le code et les mathématiques derrière cette technologie. En comprenant le processus de programmation, les concepts mathématiques s’intègrent naturellement à la démarche, permettant une bonne compréhension globale de cette technologie. Par exemple, en apprenant les interactions entre les couches de neurones, l’algèbre linéaire requise pour faire des opérations sur les matrices d’intrants et de sortants des neurones découle logiquement. De plus, nous considérerons les applications concrètes de cette technologie, ainsi que les ramifications potentielles de l’apprentissage machine dans le futur, et comment celles\sphinxhyphen{}ci devraient être utilisées afin d’améliorer la société.

D’abord, afin d’établir une base solide et d’assurer la compréhension du sujet par le lecteur, nous allons commencer par définir les notions de base du réseau neuronal. Pour ce faire, nous allons d’abord expliquer le fonctionnement général d’un programme d’intelligence artificielle qui effectue de la reconnaissance optique de caractères, ou ROC. En bref, ce type de programme transforme une image de texte, soit écrit à la main ou à l’ordinateur, en fichier de texte. Une fois cet exemple établi, nous définirons les différents termes, comme neurone, couche et expliquerons leurs différentes caractéristiques. Nous ferons également un lien entre la structure d’un réseau neuronal et celle du cerveau humain, afin de rapprocher le concept abstrait d’un réseau fait de code à quelque chose de plus tangible et connu. \sphinxcite{biblio_commented:ubiquity}

Nous allons expliquer le déroulement étape par étape d’un apprentissage machine. À l’aide d’un programme préalablement codé ayant pour objectif de reconnaître un chiffre écrit. Nous allons donc utiliser et expliquer les étapes de ce programme afin de mieux comprendre les réseaux neuronaux. Le  traitement des données et comment celles\sphinxhyphen{}ci sont introduites dans le programme sera expliqué. Ensuite, le concept de “backpropagation”, qui correspond au fait d’entraîner un programme en fournissant les entrants et les extrants désirés, sera divisé afin de mieux le comprendre.  Parmi ces divisions se retrouvera le concept de la fonction du coût qui correspond au niveau de réussite du programme non entraîné. Ainsi ce résultat de coût impactera les différents paramètres modifiables mentionnés plus tôt. La modification de ces paramètres sera expliquée et démontrée. Enfin, une démonstration de l’efficacité du programme et les impacts de ces résultats.\sphinxcite{biblio_commented:michael}

Afin de déterminer la manière dont l’intelligence artificielle doit être utilisée pour être bénéfique pour l’être humain, nous allons comparer ses avantages et inconvénients. Pour ce qui est question des bienfaits, nous comptons étudier les diverses tâches pouvant exclusivement être accomplies par l’intelligence artificielle. Par exemple, nous comptons traiter du fait que plusieurs problèmes sont tout simplement impossibles à résoudre à l’aide de raisonnement seul ou d’application de formules mathématiques, tandis que l’IA peut s’entraîner des milliers de fois à résoudre un problème spécifique, dépassant largement les capacités des autres méthodes. Pour les inconvénients, nous pensons parler du biais inhérent à la nature d’une intelligence artificielle. Puisque cette dernière apprend à l’aide d’une base de données, la qualité et la diversité de celle\sphinxhyphen{}ci influencent grandement le degré de précision de résultats pour chaque type de donné. Nous allons explorer les enjeux éthiques des problèmes de discrimination qui peuvent en résulter.


\section{Plan de rédaction du travail}
\label{\detokenize{plan:plan-de-redaction-du-travail}}\label{\detokenize{plan::doc}}
Nous allons répondre à notre question sous plusieurs aspects;


\subsection{Introduction}
\label{\detokenize{plan:introduction}}\begin{itemize}
\item {} 
Recherche et documentation sur l’histoire de l’intelligence ainsi que ses
utilités dans le passé.

\item {} 
L’intelligenge artificielle présentement

\item {} 
Formulation de l’hypothèse

\end{itemize}


\subsection{Notion de base d’un réseau neuronal}
\label{\detokenize{plan:notion-de-base-d-un-reseau-neuronal}}\begin{itemize}
\item {} 
Définition de OCR

\item {} 
Explications des neurones
\begin{itemize}
\item {} 
Structure d’un neurone

\item {} 
poids

\item {} 
biais

\item {} 
entrants et extrants

\end{itemize}

\item {} 
Communication entre les couches
\begin{itemize}
\item {} 
Fonction d’activation (démonstration mathématique)

\item {} 
lien avec les neurones biologiques

\end{itemize}

\end{itemize}


\subsection{Apprentissage machine (explication à l’aide d’un programme)}
\label{\detokenize{plan:apprentissage-machine-explication-a-l-aide-d-un-programme}}\begin{itemize}
\item {} 
Explication de la collecte et traitement de données
\begin{itemize}
\item {} 
Préparation des données

\end{itemize}

\item {} 
Fonction de coût
\begin{itemize}
\item {} 
Exemples de programmation

\item {} 
Démonstration mathématique

\end{itemize}

\item {} 
Explication de la modification des paramètres
\begin{itemize}
\item {} 
Démonstration mathématique / programmation

\end{itemize}

\item {} 
Démonstration du programme

\end{itemize}


\subsection{Impact de l’intelligence artificielle}
\label{\detokenize{plan:impact-de-l-intelligence-artificielle}}\begin{itemize}
\item {} 
Bienfaits

\item {} 
Inconvénients

\end{itemize}


\subsection{Conclusion}
\label{\detokenize{plan:conclusion}}\begin{itemize}
\item {} 
Retour sur notre hypothèse

\item {} 
Réponse à la question

\item {} 
Ouverture

\end{itemize}


\section{Bibliographie Commentée}
\label{\detokenize{biblio_commented:bibliographie-commentee}}\label{\detokenize{biblio_commented::doc}}

\subsection{Brookings, How artificial intelligence is transforming the world}
\label{\detokenize{biblio_commented:brookings-how-artificial-intelligence-is-transforming-the-world}}
\sphinxcite{biblio_commented:brookings}

Cette source est vraiment intéressante à utiliser, car elle englobe une
grande partie de l’information sur l’intelligence artificielle. Elle est donc très utile pour notre
travail, car elle parle des qualités de l’Intelligence Artificielle, l’application de
celle\sphinxhyphen{}ci dans différents secteurs, les régulations, la politique et les problèmes éthiques de
l’intelligence artificielle. Cette source est donc importante à la compréhension de l’utilité
de l’intelligence artificielle et ses enjeux. Elle est de plus très pertinente pour l’énoncé du sujet,
ou l’on parle plus globalement de l’intelligence artificielle. La section sur les secteurs appliqués
ainsi que celle sur les enjeux éthiques vont être très importantes pour expliquer la problématique
ainsi que les évenements relié a l’intelligence artificielle.


\subsection{Ministère de l’Économie et de l’Innovation, Les avantages et inconvénients de l’intelligence artificielle}
\label{\detokenize{biblio_commented:ministere-de-l-economie-et-de-l-innovation-les-avantages-et-inconvenients-de-l-intelligence-artificielle}}
\sphinxcite{biblio_commented:gouvqc}

Cette source est importante pour notre projet en plus d’être fiable,
car elle provient de notre gouvernement. Grâce à cette source, nous pourrons développer d’avantage
sur les conséquences de l’implementation de l’intelligence artificielle dans plusieurs secteurs.
Elle est donc très importante pour notre travail, pour ainsi développer sur ces aspects lors de l’énoncé du sujet.
Cette source vient complémenter celle de Brookings: \sphinxhref{https://www.brookings.edu/research/how-artificial-intelligence-is-transforming-the-world/\#\_edn4}{How artificial intelligence is transforming the world}
avec plus d’informations. Nous utiliserons les secteurs affectés par l’intelligence artificielle ainsi
que les risques liés à l’intelligence artificielle ce qui reflète la problématique de comment devrait\sphinxhyphen{}on
utiliser l’intelligence artificielle pour bénéficier l’être humain.


\subsection{Ubiquity, HUMAN BRAIN AND NEURAL NETWORK BEHAVIOR, A COMPARISON}
\label{\detokenize{biblio_commented:ubiquity-human-brain-and-neural-network-behavior-a-comparison}}
\sphinxcite{biblio_commented:ubiquity}

Cet article provient de la revue scientifique revue par les pairs de l’ACM, soit Association for Computing Machinery.
Il est donc possible d’affirmer qu’elle est très fiable et pertinente. Dans cette article, l’auteur compare
les caractéristiques d’un réseau neuronal à celles du cerveau humain. Il aborde par exemple comment les deux apprenent et
se développent avec le temps, en commençant de zéro, comme un bébé. Ce genre de comparaison nous permet d’accomplir notre objectif
pour cette section, soit de rendre le concept de réseau neuronal plus tangible et facile à assimiler.


\subsection{Michael A. Nielsen, Neural Network and Deel Learning}
\label{\detokenize{biblio_commented:michael-a-nielsen-neural-network-and-deel-learning}}
\sphinxcite{biblio_commented:michael}

Cette source est intéressante puisqu’elle explique en profondeur le fonctionnement
d’un réseau neuronal en donnant un exemple débutant par la reconnaissance optique
de caractères qui est un exemple du programme semblable à celui que nous allons
écrire. De plus, de nombreuses démonstrations mathématiques imagées. Également
cette source donne des astuces afin de créer notre programme. Cette source est
très détaillée. De plus, l’auteur est un scientifique qui a aidé à débuter
l’informatique quantique en plus de grandement s’intéresser à l’intelligence
artificielle. Malgré tout, cette source est en anglais ce qui peut
légèrement nuire à la compréhension. Nous allons principalement utiliser la
section concernant un exemple d’utilisation de l’intelligence
artificielle afin de reconnaître des chiffres écrits à la main.


\subsection{Face Recognition Vendor Test (FRVT) Part 3: Demographic Effects}
\label{\detokenize{biblio_commented:face-recognition-vendor-test-frvt-part-3-demographic-effects}}
\sphinxcite{biblio_commented:nistbias}

Cette publication gouvernementale semble fiable. Elle a été citée par de
multiples articles ultérieurement consultés. L’organisme derrière la
publication, le National Institute of Standards and Technology, est un organisme
fondé en 1901 afin de facilité la standardisation des procédés entre les
entreprises et les organisations gouvernementales de développement technologique
aux États\sphinxhyphen{}Unis.

Cette publication est aussi utile puisqu’elle consiste des résultats d’une
recherche sur les imprécisions de la reconnaissance faciale à l’aide de
l’intelligence artificielle. Le sujet sera abordé au courant du rapport et
les statistiques contenues dans cet article du NIST seront utilisées pour
supporter plusieurs arguments.

Finalement, il est aussi intéressant d’avoir une publication du NIST pour
supporter la recherche puisque nous comptons utiliser des données publiées
par cette même organisation pour l’entraînement de notre programme.



\begin{sphinxthebibliography}{Innovati}
\bibitem[InnovationQuebec, 2018]{biblio_commented:gouvqc}
et Innovation Québec, É. (2018). \sphinxstyleemphasis{Les avantages et inconvénients de l’intelligence artificielle}.
\bibitem[Grother et al., 2019]{biblio_commented:nistbias}
Grother, P., Ngan, M., \& Hanaoka, K. (2019). \sphinxstyleemphasis{Face Recognition Vendor Test (FRVT) Part 3: Demographic Effects}. NIST.
\bibitem[LAURO, n.d.]{biblio_commented:ubiquity}
LAURO, D. M. (n.d.). \sphinxstyleemphasis{HUMAN BRAIN AND NEURAL NETWORK BEHAVIOR A COMPARISON}.
\bibitem[Metz, 2020]{biblio_commented:cnnportland}
Metz, R. (2020). \sphinxstyleemphasis{Portland passes broadest facial recognition ban in the US}.
\bibitem[Nielsen, 2019]{biblio_commented:michael}
Nielsen, M. A. (2019). \sphinxstyleemphasis{Neural Networl and Deep Learning}.
\bibitem[West \& Allen, 2018]{biblio_commented:brookings}
West, D. M., \& Allen, J. R. (2018). \sphinxstyleemphasis{How artificial intelligence is transforming the world}.
\end{sphinxthebibliography}







\renewcommand{\indexname}{Index}
\printindex
\end{document}